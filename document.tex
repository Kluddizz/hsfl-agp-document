\documentclass[sigconf]{acmart}
\title{Simulation von Protuberanzen und Erstellen von prozeduralen Kugeloberflächen in Unity}
\author{Florian Hansen}

\affiliation{
  \institution{Hochschule Flensburg}
}

\usepackage[ngerman]{babel}

\settopmatter{printacmref=false}
\renewcommand\footnotetextcopyrightpermission[1]{}
\pagestyle{plain}

\begin{document}
  \begin{abstract}
    test
  \end{abstract}
  \maketitle

  \section{Einleitung}
  \section{Homogene Kugeloberflächen}
  \subsection{Das Problem mit naiven Methoden}
  \subsection{Fibonacci-Spiralen}
  \subsection{Ikosaeder}
  \subsection{Entwicklung eines Skripts}
  \section{Sonnenoberfläche}
  \subsection{Generierung des Meshes} 
  \subsection{Fractal Brownian Motion}
  \subsection{Cellular Noise}
  \subsection{Entwicklung eines Shaders}
  \section{Protuberanzen}
  \subsection{Wie Sonnenstürme entstehen}
  \subsection{Magnetische Felder am Beispiel eines Dipols}
  \subsection{Erstellung eines Vektorenfelds}
  \subsection{Aufbau eines Partikelsystems}
  \section{Fazit und Ausblick}
\end{document}
