\documentclass[sigconf]{acmart}
\title{Simulation von Protuberanzen und Erstellen von prozeduralen Kugeloberflächen in Unity}
\author{Florian Hansen}

\affiliation{
  \institution{Hochschule Flensburg}
  \country{Germany}
}

\usepackage[ngerman]{babel}
\usepackage{tabularx}
\usepackage{placeins}
\usepackage[ruled,vlined,linesnumbered,german]{algorithm2e}

\settopmatter{printacmref=false}
\renewcommand\footnotetextcopyrightpermission[1]{}

\begin{document}
  \begin{abstract}
    Das Simulieren von echten Phänomenen kann Menschen dabei helfen
    bestimmte Vorgänge in der Natur besser verstehen zu können. Aber
    nicht nur das Verstehen spielt eine wichtige Rolle in der Wissenschaft,
    sondern auch die Validierung aufgestellter Modelle. Diese Arbeit widmet
    sich der Simulation der Sonne. Zum einen wird erforscht, wie die 
    Oberfläche generiert werden kann, zum anderen sollen Protuberanzen, also
    Sonnenstürme, simuliert werden. Diese Arbeit zeigt, wie die komplexen
    Phänomene der Sonne vereinfacht und in Echtzeit dargestellt werden
    können.
  \end{abstract}
  \maketitle
  \pagestyle{plain}

  \section{Einleitung}
Häufig steht man als 3D-Programmierer vor dem Problem, in der Natur vorkommende
Phänomene effizient und in Echtzeit darzustellen, um diese dann anschließend in
andere Projekte wie Videospiele einzubinden. Aber auch bei Filmen sollte darauf
geachtet werden, dass die Zeit für das Rendern nicht ausartet. Im Grunde
existieren zwei Lösungsansätze für das Problem, rechenintensive
Echtzeitsimulationen durchzuführen. Zum einen kann man fiktive Effekte
definieren, die durch iteratives Ausprobieren ähnliche Ergebnisse wie das reale
Vorbild erzielen. Zum anderen kann man versuchen, physikalische Prozesse
abzubilden, um so nah an der Realität, wie möglich zu bleiben. Diese
wissenschaftliche Arbeit soll sich diesem Problem anhand der uns bekannten Sonne
widmen. Hierbei wird sich auf die äußerliche Erscheinung der Sonne und besonders
ihrer Protuberanzen (umgangssprachlich Sonnenstürme) bezogen. Es soll also eine
Verbindung zwischen echten, beobachtbaren Phänomenen bzw. physikalischen
Eigenschaften der Sonne und Computersimulationen umgesetzt werden.

  \section{Berechnung einer Kugel}
Damit später eine Textur für die Sonne erstellt werden kann, muss zuerst eine
Oberfläche generiert werden, die diese Textur aufnehmen kann. Hierfür existieren
verschiedenste Möglichkeiten. Im Falle der Sonne betrachten wir das Modell einer
Kugel, die später manipuliert werden soll, um visuelle Effekte auf ihr
abzubilden. Beim Betrachten der folgenden Algorithmen zur Erzeugung von
Kugeloberflächen wird klar, dass nicht alle Methoden für dieses Vorhaben
geeignet sind. Aus diesem Grund werden die einzelnen Verfahren näher erläutert
und anschließend abgewogen, welcher Algorithmus zur Erzeugung einer
Sonnenoberfläche verwendet werden sollte.

\subsection{Die naive Methode}
Die einfachste Möglichkeit, eine Kugeloberfläche zu erzeugen, ist die Verwendung
von Segmentlinien. Im Grunde unterteilt man die gesamte Kugel in Schichten,
deren Umfang schließlich die Oberfläche der Kugel darstellen. Die Mesh-Punkte
(\textit{Vertices}) dieser Schichten, welche als einfache Kreise wahrgenommen
werden können, werden mithilfe von Längen- und Breitengradinformationen
erstellt. Das Resultat ist eine Kugeloberfläche mit erkennbaren Polen an der
Ober- und Unterseite der Oberfläche. Diese entstehen dadurch, dass auch am Rand
der Kugel die gleiche Anzahl an Kanten vorhanden ist, wie am Äquator. Diese
Schichten sind dementsprechend lediglich herunterskaliert und in einem immer
kürzer werdenden Abstand aufeinander geschichtet und Erzeugen eine Überabtastung
am oberen und unteren Rand der Kugel.

Die Überabtastung an den Polen bzw. die Unterabtastung am Äquator bringt viele
Probleme mit sich. Vor allem, wenn die Oberfläche animiert werden soll,
erscheint die Animation als sehr unregelmäßig. Hierauf wird jedoch in einem
späteren Abschnitt eingegangen, wenn die hier vorgestellten Methoden miteinander
verglichen werden.

\subsection{Fibonacci-Spiralen}
\subsection{Ikosaeder}
\subsection{Entwicklung eines Skripts}

  \section{Sonnenoberfläche}
In diesem Abschnitt wird besprochen, wie die Sonnenoberfläche in der Realität
aussieht und wie die Erscheinung prodedural Nachgebildet werden kann. Dabei
wird insbesondere auf die Entwicklung eines entsprechenden CG-Shaders
eingengangen. Grundsätzlich besteht die Möglichkeit, eine Textur zu
generieren, diese in eine Datei zu speichern und anschließend auf das Mesh zu
übertragen. Dies hat jedoch den Nachteil, dass die Textur nur unter sehr
hohem Aufwand über die Zeit verändert werden kann. Deshalb soll die
Oberflächentextur im Shader selbst erzeugt werden, sodass diese in Echtzeit
animiert werden kann. Auch wird das Verfahren erläutert, wie die
Sonnengeometrie erzeugt wird und für den Shader vorbereitet wird.

\subsection{Generierung der Geometrie} 
Bevor eine Textur für die Sonne dargestellt werden kann, müssen zuerst alle
Rahmenbedingungen stimmen. Dies beinhaltet vor allem die Vorbereitung eines
geeigneten Meshes für die Sonnenoberfläche. Im Folgenden wird der Algorithmus
vorgestellt, welcher eine homogene Geometrie für die Sonne erstellt. Dieser
erstellt in erster Linie die Eckpunkte einer Icosphere. Eine Icosphere
besitzt 20 Dreiecke bestehend aus 12 Eckpunkten, welche die Grundlage für
weitere Unterteilungsschritte sind. Dementspreched müssen diese initial
berechnet werden. Anschließend startet der eigentliche Algorithmus zum
Unterteilen der Flächen in kleinere Flächen. Dieser berechnet auf jeder Kante
eines Dreiecks einen Mittelpunkt, sodass für jede Dreiecksfläche insgesamt
drei neue Eckpunkte entstehen. Anschließend müssen die neu erzeugten Punkte
auf die Einheitskugel abgebildet werden, indem die Positionsvektoren
normalisiert werden. Algorithmus \ref{alg:icosphere} soll diesen Vorgang
verdeutlichen.

\begin{algorithm}
  \caption{Unterteilen von Dreiecksflächen auf einer Icosphere}
  \label{alg:icosphere}
  \SetAlgoLined

  \SetKwInOut{Input}{Eingabe}\SetKwInOut{Output}{Ausgabe}

  \Input{Flächen des Basismeshes $M$, Rekursionslevel $n$}
  \Output{Mesh einer Icosphere mit unterteilten Flächen}
  \BlankLine
  \For{$i\leftarrow 0$ \KwTo $n$}{
    $M'\leftarrow \{\}$\;
    \BlankLine
    \ForEach{$d \in M$}{
      $(\vec{a},\;\vec{b},\;\vec{c})\leftarrow d$\;
      \BlankLine
      $\vec{a}'\leftarrow$ calculateMiddlePoint($\vec{a}$, $\vec{b}$)\;
      $\vec{b}'\leftarrow$ calculateMiddlePoint($\vec{b}$, $\vec{c}$)\;
      $\vec{c}'\leftarrow$ calculateMiddlePoint($\vec{c}$, $\vec{a}$)\;
      \BlankLine
      append($M'$, $(\vec{a},\;\vec{a}',\;\vec{c}')$)\;
      append($M'$, $(\vec{b},\;\vec{b}',\;\vec{a}')$)\;
      append($M'$, $(\vec{c},\;\vec{c}',\;\vec{b}')$)\;
      append($M'$, $(\vec{a}',\;\vec{b}',\;\vec{c}')$)\;
    }
    \BlankLine
    $M\leftarrow M'$\;
  }
\end{algorithm}

\begin{figure}
  \includegraphics[width=\columnwidth]{icosphere-algorithm}
  \caption{Vergleich zwischen unterschiedlichen Rekursionslevel von Algorithmus \ref{alg:icosphere}. Links befindet sich das Basismodell, in der Mitte die erste und rechts die zweite Rekursionsstufe.}
  \label{fig:icosphere-levels}
  \Description[]{}
\end{figure}

\subsection{Fractal Brownian Motion}
Eine Textur zu erzeugen, die zu jeden Zeitpunkt gleich aussieht wirkt oft
langweilig und uninteressant. Auch, wenn man das Vorbild dieser Arbeit
betrachtet -- die Sonne --, unterscheidet sich die Oberfläche zu jedem
Zeitpunkt von einer vorherigen Messung. Um ein solchen Phänomen zu
modellieren, bedient man sich häufig dem Rauschen. Rauschen hat für
unterschiedliche Fachgebiete eine unterschiedliche Bedeutung. Musiker
verbinden damit störende Geräusche und Astrophysiker denken dabei an
kosmische Hintergrundstrahlung \cite{bookofshaders}. In der Computergrafik
versucht man hingegen, dieses Rauschen künstlich zu generieren, um
beispielsweise eine Textur prozedural zu erzeugen. Auch in dieser Arbeit soll
die Oberflächentextur der Sonne durch ein künstliches Rauschen erzeugt und
verändert werden. Dies entspricht natürlich nicht der Realität, denn die
Sonnenoberfläche hängt von vielen Faktoren, wie der Temperatur und dem
Magnetfeld der Sonne ab. Auch kann man sogenannte Sonnenflecken beobachten,
welche sich als dunkle Flecken auf der Oberfläche äußern. Diese sind nicht
wirklich schwarz, sondern erscheinen dunkler als ihre Umgebung, da diese
Stellen niedrigere Temperaturen aufweisen.

Anstatt nun zu versuchen, die Oberfläche der Sonne so realitätsgetreu wie
möglich nachzubilden, wird in diesem Projekt versucht, mithilfe von künstlich
erzeugtem Rauschen eine ähnliche Oberfläche zu generieren. Hierfür wird, wie
der Titel des Abschnitts andeutet, \textit{Fractal Brownian Motion} verwendet.
Oft werden hierbei Begriffe wie \textit{Oktaven}, \textit{Porosität} und
\textit{Zuwachs} genannt. Eine Oktave beschreibt eine Summe aus mehreren
Rauschfunktionen, die Porosität die Multiplikation der Frequenz um einen konstanten
Wert und der Zuwachs die Verringerung der Amplitude. Algorithmus \ref{alg:fbm} zeigt
ein einfaches zweidimensionales frakturiertes Rauschen basierend auf \cite{bookofshaders}.

\begin{algorithm}
  \caption{Fractal Brownian Motion im zweidimensionalen Raum}
  \label{alg:fbm}
  \SetAlgoLined

  $octaves\leftarrow 1$\;
  $lacunarity\leftarrow 2$\;
  $gain\leftarrow 0.5$\;
  \BlankLine
  $amplitude\leftarrow 0.5$\;
  $frequency\leftarrow 1$\;
  \BlankLine
  \For{$i\leftarrow 0$ \KwTo $octaves$}{
    $y = y + \text{amplitude} * \text{noise}(\text{frequency} \cdot x)$\;
    $\text{frequency} = \text{frequency} \cdot \text{lacunarity}$\;
    $\text{amplitude} = \text{amplitude} \cdot \text{gain}$\;
  }
\end{algorithm}

\subsection{Entwicklung eines Shaders}

  \section{Protuberanzen}\label{sec:prominences}
Ein weiterer wesentlicher Bestandteil dieser Arbeit besteht darin, die
Protuberanzen der Sonne, umgangsprachlich Sonnenstürme, zu simulieren. Anders
als bei der Sonnenoberfläche und ihrer Textur wird hier versucht, diese
anhand von physikalischen Eigenschaften der Sonne zu berechnen. Ferner sollen
Magnetfelder berechnet werden, die Einfluss auf Partikel der Sonne nehmen.
Dadurch sollen Protuberanzen entstehen und simuliert werden.

Eine Protuberanz ist ein auf der Sonnenoberfläche auftretendes Phänomen. Sie
entstehen durch abprupte Neuverbindungen von Magnetfeldlinien. Dies wird auch
als Rekonnexion bezeichnet und geschieht immer dann, wenn zwei sich zwei
Magnetfelder aufeinander zu ausbreiten und sich dann neu verbinden
\cite{SpanierReconnexion}. Dabei wird eine große Menge Energie freigesetzt,
die dann Material aus der Sonne hinausschleudert \cite{ZirinProminences}. Da
die Sonne vermutlich nicht nur ein einfaches Magnetfeld besitzt, wäre eine
vollständige Simulation aller Magnetfelder bzw. Pole nicht sinnvoll, da dies
ab einer bestimmten Anzahl von Magnetfeldern einen zu hohen Rechenaufwand
bedeuten würde und man die Phänomenen nicht in Echtzeit darstellen könnte.
Deshalb wird sich im folgenden auf das Modell eines Dipols gestützt, welches
einfach zu berechnen und trotzdem realitätsnahe Ergebnisse erzielen kann. Ein
Dipol ist im Prinzip ein einfacher Magnet und besteht aus zwei
unterschiedlichen Ladungen, anhand dessen ein Magnetfeld berechnet werden
kann. Der Dipol ist das einfachste Modell, welches sich mit Magnetismus
beschäftigt. Die magnetische Flussdichte $\vec{B}$ eines Dipols wird durch
folgende Gleichung beschrieben \cite{Stoecker2013}.
\[
  \vec{B}(\vec{r}) = \frac{\mu_0}{4\pi r^2} \frac{3\vec{r}(\vec{m} \cdot
  \vec{r}) - \vec{m}r^2}{r^3}
\]
Dabei beschreibt $\vec{r}$ den Positionsvektor des zu betrachtenden Punktes
und $r = \vert\vec{r}\vert$ die Länge von $\vec{r}$, $\mu_0$ die magnetische
Feldkonstante und $\vec{m} = p(\vec{r}_+ - \vec{r}_-)$, wobei $p$ die
Polstärke, $\vec{r}_+$ die Position vom Pluspol und $\vec{r}_-$ die Position
vom Minuspol sind \cite{brown1962magnetostatic}. Mithilfe dieser Gleichung
lassen sich für alle Punkte in dem Magnetfeld eines Dipols die entsprechenden
Kräfte berechnen.

\subsection{Vektorenfelder und Partikel}
Das Problem bei der Simulation von Wolken, Rauch, Feuer und auch von
Protuberanzen ist, dass diese Gebilde nicht wohlgeformt sind, wie
beispielsweise ein Würfel, eine Kugel oder sogar ein 3D-Modell eines
Gebäudes. Anstatt zu versuchen Rauch mithilfe von starren Körpern
darzustellen, kam man auf die Idee, Partikelsysteme zu definieren. Partikel
repräsentieren dabei keine einzelnen Objekte, sondern haben die Aufgabe,
Objekte aufgrund ihrer Eigenschaften selbst zu formen. Dabei sind die Partikel
keine statischen Entitäten, sondern üben Kräfte untereinander aus und erfahren
Kräfte von außen, die beispielsweise eine Bewegung der einzelnen Partikel
hervorrufen. Ferner besitzen Partikel bestimmte Eigenschaften wie z.B. eine
Lebensdauer, wodurch sie \textit{geboren} werden, aber auch \textit{sterben}
können. Weitere übliche Eigenschaften sind Geschwindigkeit, Position, Größe,
Farbe und Form. Ein Objekt, welches durch ein Partikelsystem dargestellt
wird, ist nicht deterministisch, da seine Form nicht komplett beschrieben
wird. Stattdessen werden stochastische Methoden angewandt, um das Aussehen
dieser Objekte zu verändern \cite{Reeves1983}.

Auch im Falle der Sonne, um die es in dieser Arbeit handelt, können
Partikelsysteme dabei helfen, komplexe Phänomene realitätsnah darzustellen,
anstatt sie in vorherigen Arbeitsprozessen statisch zu modellieren. In diesem
Projekt werden Partikelsysteme von Unity verwendet. Genauer gesagt werden
\textit{Visual-Effect-Graphen} definiert, welche die Partikel und Kräfte, die
auf sie einwirken, kontrollieren. Damit sollen vor allem versucht werden, die
auf der Sonne auftretenden Protuberanzen zu simulieren. Hierfür wird der im
Abschnitt \ref{sec:prominences} kurz besprochene Dipol verwendet werden. Die
Idee ist, nicht alle Magnetfelder der Sonne abzubilden, Rekonnexionen ihrer
Magnetfeldlinien zu berechnen und damit Protuberanzen zu erzeugen, sondern
pro Protuberanz einen Dipol zu definieren, welcher die Form eines solchen
Materiestroms beeinflusst bzw. hervorruft. Um nun ein Magnetfeld als
Krafteinwirkung auf Partikel zu erstellen, wird ein Vektorenfeld aufgebaut
und anschließend als 3-dimensionale Textur gespeichert. Diese Textur wird
dann in das Partikelsystem eingebaut und beeinflusst damit die Bewegung und
Geschwindigkeit der Partikel.

Ein Vektorfeld ist eine Sammlung von Vektoren, die jeweils eine Position
und eine Richtung besitzen. Damit erfährt jedes Objekt, welches sich an einer
bestimmten Stelle innerhalb des Vektorfeld befindet, eine Kraft in der
jeweiligen Richtung. Um ein solches Feld zu erzeugen,
werden Vektoren gleichmäßig im Raum verteilt. Anschließend werden die Richtungen
mithilfe der magnetischen Flussdichten für jeden dieser Vektoren bestimmt, sodass
sich diese anhand des Magnetfelds ausrichten. Jetzt, wo Vektoren im Raum
verteilt und ihre Richtungen definiert wurden, wird eine 3D-Textur erzeugt, die
all diese Informationen speichert. Dabei wird die Position eines Vektors durch
den Index des Pixels und die Richtung durch eine Farbe innerhalb der Textur
repräsentiert. Abbildung \ref{fig:dipole-vector-field-unity} stellt dies in einem
zweidimensionalen Raum dar. Die damit erstellte 3D-Textur bestehend aus
Richtungsvektoren, die von einer Position abhängen, kann nun innerhalb eines
Partikelsystems verwendet werden. Damit die Partikel die Form einer Protuberanz
annehmen, müssen deren Eigenschaften entsprechend justiert werden. Beispielsweise
muss die initiale Position der Partikel innerhalb der Sonne sein. Dann muss eine
explosionsähnliche Kraft auf sie einwirken, die die Partikel nach außen treibt.
Diese Kraft simuliert damit die Rekonnexion von Magnetfeldlinien und steht senkrecht
auf der Sonnenoberfläche und zeigt nach außen. Da das Magnetfeld ebenfalls vorhanden
ist, werden sich die Partikel entlang der Magnetfeldlinien bewegen. Das Ergebnis
sind Protuberanzen, die mithilfe eines Dipols simuliert und in Abbildung \ref{fig:flare}
betrachtet werden können.

\begin{figure}
  \includegraphics[width=0.75\columnwidth]{dipole-vector-field-unity}
  \caption{In Unity berechnete Magnetfeldlinien des Dipol-Modells. Der Einfachheit halber wird hier ein 2D-Vektorenfeld dargestellt.}
  \label{fig:dipole-vector-field-unity}
  \Description[Dipole]{Magnetfeldlinien eines Dipols.}
\end{figure}

\begin{figure}
  \includegraphics[width=\columnwidth]{flare}
  \caption{Kraftauswirkung eines Dipol-Magnetfeldes auf Partikel der Sonne. Das Ergebnis sind vereinfachte Protuberanzen.}
  \label{fig:flare}
  \Description[Protuberanzen]{Kraftauswirkung eines Dipol-Magnetfeldes auf Partikel der Sonne. Das Ergebnis sind vereinfachte Protuberanzen.}
\end{figure}
  \section{Fazit und Ausblick}
In dieser Arbeit wurde besprochen, wie eine Sonne mithilfe von prozeduralen
Algorithmen simuliert werden kann. Dabei wurde explizit auf verschiedene
Möglichkeiten zum Generieren von Kugeloberflächen eingegangen und warum
inhomogene Eckpunkte auf einer solchen Kugel problematisch sind. Es wurde
sich für eine Icosphere entschieden, da dort Eckpunkte gleichmäßig verteilt
auf einer Kugeloberfläche liegen und das Fibonacci-Gitter einige zusätzliche
Probleme, wie beispielsweise ein Loch im Mesh nach einer
Delaunay-Triangulation, bereitet. Anschließend wurde ein Algorithmus zum
Berechnen von prozeduralen Texturen mithilfe von \textit{Fractal Brownian
Motion} erläutert, auf die Sonnenoberfläche übertragen und die Ergebisse
dargestellt. Zum Schluss wurde der Dipol, sein Magnetfeld und magnetische
Flussdichte kurz besprochen und die Theorie in ein Vektorenfeld übertragen.
Dieses wurde dann als Kraftfeld in ein Partikelsystem übertragen, sodass sich
die entsprechenden Partikel wie Protuberanzen der Sonne verhalten.

Zukünfig wäre es sinnvoll, den Dipol durch komplexere Modelle auszutauschen,
die ein Magnetfeld hervorrufen. Dadurch könnte eine noch realistischere
Simulation durchgeführt werden. Auch könnte untersucht werden, wie sich
verschiedene Dipole beeinflussen und wie die daraus entstehenden
Magnetfelder verwendet werden können. Darauf basieren wäre es ebenfalls
interessant, wie Rekonnexionen und deren Energiefreigabe zur Simulation
von Proutberanzen verwendet werden können.

  \bibliography{literature}
  \bibliographystyle{abbrv}
\end{document}
