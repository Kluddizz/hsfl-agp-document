\section{Einleitung}
Häufig steht man als 3D-Programmierer vor dem Problem, die in der Natur
vorkommenden Phänomene effizient und in Echtzeit darzustellen zu wollen, um
diese dann anschließend in andere Projekte wie Videospiele einzubinden. Aber
auch bei Filmen sollte darauf geachtet werden, dass die Zeit für das Rendern
nicht ausartet. Am interessantesten erscheint in diesem Zusammenhand die Nutzung
von Echtzeitsimulationen, um natürliche Prozesse näher untersuchen zu können,
die nicht so häufig vorkommen.

Im Grunde existieren zwei Lösungsansätze für das Problem, rechenintensive
Echtzeitsimulationen durchzuführen. Zum einen kann man fiktive Effekte
definieren, die durch iteratives Ausprobieren ähnliche Ergebnisse wie das reale
Vorbild erzielen. Diese Ergebnisse sind dann jedoch rein fiktiv und haben selten
etwas mit der Realität zu tun. Zum anderen kann man versuchen, physikalische
Prozesse abzubilden, um so möglichst nah an der Realität zu bleiben. Diese
wissenschaftliche Arbeit soll sich diesem Problem anhand der uns bekannten Sonne
widmen. Hierbei wird sich auf die äußerliche Erscheinung der Sonne und besonders
ihrer Protuberanzen (umgangssprachlich Sonnenstürme) bezogen. Es soll also eine
Verbindung zwischen echten, beobachtbaren Phänomenen bzw.  physikalischen
Eigenschaften der Sonne und Computersimulationen umgesetzt werden.
