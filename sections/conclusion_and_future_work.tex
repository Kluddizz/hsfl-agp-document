\section{Fazit und Ausblick}
In dieser Arbeit wurde besprochen, wie eine Sonne mithilfe von prozeduralen
Algorithmen simuliert werden kann. Dabei wurde explizit auf verschiedene
Möglichkeiten zum Generieren von Kugeloberflächen eingegangen und warum
inhomogene Eckpunkte auf einer solchen Kugel problematisch sind. Es wurde
sich für eine Icosphere entschieden, da dort Eckpunkte gleichmäßig verteilt
auf einer Kugeloberfläche liegen und das Fibonacci-Gitter einige zusätzliche
Probleme, wie beispielsweise ein Loch im Mesh nach einer
Delaunay-Triangulation, bereitet. Anschließend wurde ein Algorithmus zum
Berechnen von prozeduralen Texturen mithilfe von \textit{Fractal Brownian
Motion} erläutert, auf die Sonnenoberfläche übertragen und die Ergebisse
dargestellt. Zum Schluss wurde der Dipol, sein Magnetfeld und magnetische
Flussdichte kurz besprochen und die Theorie in ein Vektorenfeld übertragen.
Dieses wurde dann als Kraftfeld in ein Partikelsystem übertragen, sodass sich
die entsprechenden Partikel wie Protuberanzen der Sonne verhalten.

Zukünfig wäre es sinnvoll, den Dipol durch komplexere Modelle auszutauschen,
die ein Magnetfeld hervorrufen. Dadurch könnte eine noch realistischere
Simulation durchgeführt werden. Auch könnte untersucht werden, wie sich
verschiedene Dipole beeinflussen und wie die daraus entstehenden
Magnetfelder verwendet werden können. Darauf basieren wäre es ebenfalls
interessant, wie Rekonnexionen und deren Energiefreigabe zur Simulation
von Proutberanzen verwendet werden können.