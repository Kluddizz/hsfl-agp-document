\section{Berechnung einer Kugel}
Damit später eine Textur für die Sonne erstellt werden kann, muss zuerst eine
Oberfläche generiert werden, die diese Textur aufnehmen kann. Hierfür existieren
verschiedenste Möglichkeiten. Im Falle der Sonne betrachten wir das Modell einer
Kugel, die später manipuliert werden soll, um visuelle Effekte auf ihr
abzubilden. Beim Betrachten der folgenden Algorithmen zur Erzeugung von
Kugeloberflächen wird klar, dass nicht alle Methoden für dieses Vorhaben
geeignet sind. Aus diesem Grund werden die einzelnen Verfahren näher erläutert
und anschließend abgewogen, welcher Algorithmus zur Erzeugung einer
Sonnenoberfläche verwendet werden sollte.

\subsection{Die naive Methode}
Die einfachste Möglichkeit, eine Kugeloberfläche zu erzeugen, ist die Verwendung
von Segmentlinien. Im Grunde unterteilt man die gesamte Kugel in Schichten,
deren Umfang schließlich die Oberfläche der Kugel darstellen. Die Mesh-Punkte
(\textit{Vertices}) dieser Schichten, welche als einfache Kreise wahrgenommen
werden können, werden mithilfe von Längen- und Breitengradinformationen
erstellt. Das Resultat ist eine Kugeloberfläche mit erkennbaren Polen an der
Ober- und Unterseite der Oberfläche. Diese entstehen dadurch, dass auch am Rand
der Kugel die gleiche Anzahl an Kanten vorhanden ist, wie am Äquator. Diese
Schichten sind dementsprechend lediglich herunterskaliert und in einem immer
kürzer werdenden Abstand aufeinander geschichtet und Erzeugen eine Überabtastung
am oberen und unteren Rand der Kugel.

Die Überabtastung an den Polen bzw. die Unterabtastung am Äquator bringt viele
Probleme mit sich. Vor allem, wenn die Oberfläche animiert werden soll,
erscheint die Animation als sehr unregelmäßig. Hierauf wird jedoch in einem
späteren Abschnitt eingegangen, wenn die hier vorgestellten Methoden miteinander
verglichen werden.

\subsection{Fibonacci-Spiralen}
\subsection{Ikosaeder}
\subsection{Entwicklung eines Skripts}
